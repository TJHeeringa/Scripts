\documentclass{article}


\usepackage[dutch]{babel}
\usepackage{graphicx}
\usepackage{palatino} % lettertype
\usepackage{hyperref}
\frenchspacing
\title{Script Officer booksales}
\author{Tjeerd Jan Heeringa}
\makeindex


\begin{document}
\begin{center}

\section*{Script Officer Booksales}
\vspace{-3mm}
\textit{Tjeerd Jan Heeringa, last modified: 2016}
\end{center}


\section*{Main tasks Officer Booksales}
\begin{itemize}
\vspace{-1mm}
\itemsep-1mm
\item Gathering of books to be sold
\item Communicate the collection of books to be sold to the booksales company
\item Distribution and sale(1st quartile) of books
\end{itemize}


\subsection*{Gathering of books to be sold}
Every professor requires some books for his or her courses. As Officer Booksales it is your job to gather all those books, so that the students have the proper books available to them.  

\subsubsection*{Contacting the teachers}

\begin{itemize} 
\vspace{-1mm}
\itemsep-1mm 
\item Go to the bachelor and master programs and gather all the teachers which teach in the next quartile with their respective course. Only gather books for courses that start in the next quartile and not for books that already run for a quartile.
\item Mail the teachers with a request to sent you the books they will use for their courses. Include a reasonable deadline in the mail and tell them that you will use the information present from last year to determine which books will be used, if they do not respond in time.
\item List the books that follow from the responses of the teachers.
\end{itemize}

\subsection*{Communicate the collection of books to be sold to the booksales company}
To ensure that the booksales company has the proper books online in their store, it is necessary to communicate that to the company. The communications go in several phases, all of which go by email. The booksales company tells you when they need to have what information. The information you sent to them has to be placed in one of several templates they deliver. 

\subsubsection*{Place order}
The first step is telling the company what you want from them.

\begin{itemize} 
\vspace{-1mm}
\itemsep-1mm 
\item Open the template for the order
\item Fill in the approximate number of students for every year
\item Fill in the main part of the sheet with the information gathered from the professors in the previous section.
\item Make an optimistic estimate of the number of books that are to be bought by the students
\item FOR FIRST QUARTILE ONLY Order a pin terminal to come along with the books
\end{itemize}

\subsubsection*{Bargain}
The second step is to respond to their counter offer. The booksales company wants to sell as many books as possible, but does not want to have any books left over after the sales have ended, so the company will always make a pessimistic counter offer based on the sales of last year to ensure they do not have any books left over. The problem with that is that if there are more students buying books then their pessimistic estimate, then some students will get their books really late, due to production times.

\begin{itemize} 
\vspace{-1mm}
\itemsep-1mm 
\item Compare their offer to yours
\item Determine whether all students that are willing to buy a book will get a book in time
\item Sent a counter offer with a better yet still optimistic estimate
\item Repeat the previous steps for every offer they sent you and change your estimate accordingly
\end{itemize}

\subsubsection*{Check}
A little while after the order has been placed, the booksales company will make the books available through their website. Do not take for granted what they place there and check properly. Do the following steps for all the study years.

\begin{itemize} 
\vspace{-1mm}
\itemsep-1mm 
\item Look at the books and compare them to the list of books gathered from the professors 
\item Compare the prices of the books to the prices the study store promised you when you placed your order 
\item OPTIONAL Compare your prices to the prices of the other associations
\item Check whether the additional information is the same as you told the booksales company
\item If something is wrong, but is not in favor of the student, tell the booksales company what is wrong and discuss a solution
\end{itemize}

\subsection*{Distribute the bought books}
Once the students bought their books, a part of the books will be delivered at Abacus. The way this happens divers in the first quartile for the rest of the quartiles. The reason for this is that the first year students buy their books on site and not in advance. For the first quartile the booksales company will deliver a number of books equal or bigger than the number of expected students. For the first quartile the books will be delivered at the start of the kick-in and for the other quartiles the books will be delivered in the first few days of the quartile.   

\subsubsection*{Book distribution in the first quartile}

\begin{itemize} 
\vspace{-1mm}
\itemsep-1mm 
\item Gather the books behind the board table in Abacus room
\item Check whether the correct amount of books has been delivered
\item Follow the manual of the pin terminal to ensure it is fully functional. Make sure to print the PIN receipt that is stored in memory of the PIN terminal beforehand
\item Split the books into books for the upcoming students and books for the rest
\item Place the books for the upcoming students and a committee brochure in bags such that each student can buy one of it buy paying with the PIN terminal
\item Plan a moment together with the Twick-in to do the booksales
\item Note each buy and archive the list in the books folder together with the receipt from the PIN terminal 
\item Once the first week of the quartile starts, place a message on Facebook to inform the students that they can pick up their books
\item place a list with the names of the rest of the students and mark everyone who came to pick up their book
\end{itemize}

\subsubsection*{Book distribution in the other quartiles}

\begin{itemize} 
\vspace{-1mm}
\itemsep-1mm 
\item Gather the books behind the board table in Abacus room
\item Check whether the correct amount of books has been delivered
\item OPTIONAL Once the first week of the quartile starts, place a message on Facebook to inform the students that they can pick up their books
\item Place a list with the names of students who bought books and mark everyone who came to pick up their book. Archive the list in the books folder after everyone picked up their books
\end{itemize}

\subsubsection*{Financial followup}
Once the booksales have finished for the quartile, an invoice will be sent to Abacus telling you the number of books that have been sold and the amount Abacus has to pay the booksales company. As Abacus is an intermediary, the amount Abacus has to pay will be zero always.  

\subsubsection*{Troubleshoot}
\begin{itemize} 
\vspace{-1mm}
\itemsep-1mm 
\item The invoice is not zero\\
In this case two things might be happening. Either someone at the booksales company messed up or a transaction has gone wrong. Both times you have to inform the booksales company and find a solution together with them.
\item Not all books are present\\
Mail the booksales company that not all books are present. Once you get a response from the booksales company with an estimate for when the books might arrive or when the booksales company takes a long time to respond, inform the student with the matter at hand.   
\end{itemize}

\end{document}